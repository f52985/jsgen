\begin{abstract}\label{sec:abstract}
  In 2015, ECMAScript 6 (ES6, 2015) introduced a new wave in JavaScript with
  various new language features. It is the start of the fast-evolving nature of
  JavaScript with an annual release cadence and open development process.
  However, most JavaScript engines did not support or partially supported newly
  introduced features in the beginning. Therefore, transpilers showed up to
  fulfill developers' desires to use new language features before JavaScript
  engines would support them. They \textit{desugar} such features to transpile
  JavaScript programs from ES6 or later versions to ECMAScript 5 (ES5, 2009).
  Even though most engines now support language features even in the latest
  ECMAScript, developers still heavily utilize them to execute their JavaScript
  programs in legacy engines. However, there is no empirical study to answer the
  correctness or quality of their desugaring process even though six years have
  passed since 2015.

  This paper takes the first step to systematically studying the desugaring
  process of the most dominating JavaScript transpiler, Babel. We measure the
  quality of desugaring process from ES6 or later versions to ES5 based on its
  \textit{semantics preservation}, \textit{speed degradation}, and \textit{code
  bloat}. To fairly measure the coverage of JavaScript language features, we
  introduce their solid definition and an automated approach to extract them
  from JavaScript programs using their statistical information. Moreover, we
  develop $\tool$, a JavaScript program generator using genetic algorithms, to
  automatically find the worst quality cases in each metric. Our experiments
  show that Babel incorrectly desugars \inred{XX} language features, degrades
  the execution speed by \inred{XX.X}\% on average, and bloats code sizes by
  \inred{XX.X}\% on average. In the worst case, the speed was \inred{XX.X} times
  slower, and the size was \inred{XX.X} times increased. We believe that this
  study provides a foundation for building sound and practical JavaScript
  transpilers.
\end{abstract}
