\begin{abstract}\label{sec:abstract}
  In 2015, ECMAScript 6 (ES6, 2015) introduced a new wave in JavaScript with
  various new language features. It is the start of the fast-evolving nature of
  JavaScript with an annual release cadence and open development process.
  However, most JavaScript engines did not support or partially supported newly
  introduced features in the beginning. Therefore, transpilers showed up to
  fulfill developers' desires to use new language features before JavaScript
  engines would support them. They \textit{desugar} such features to transpile
  JavaScript programs from ES6 or later versions to ECMAScript 5.1 (ES5.1,
  2011). Even though most engines now support language features even in the
  latest ECMAScript, developers still heavily utilize them to execute their
  JavaScript programs in legacy engines. However, there is no empirical study to
  answer the correctness or quality of their desugaring process even though six
  years have passed since 2015.

  This paper takes the first step to systematically studying the desugaring
  process of the most dominating JavaScript transpiler, Babel. We measure the
  quality of desugaring process based on its \textit{semantics preservation},
  \textit{speed degradation}, and \textit{code bloat}. Moreover, we develop
  $\tool$, a JavaScript program generator using genetic algorithms, to
  automatically find the worst quality cases in each metric.
\end{abstract}
