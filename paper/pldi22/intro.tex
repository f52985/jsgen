\section{Introduction}\label{sec:intro}

JavaScript~\cite{js-hopl} has quickly evolved with an annual release cadence and
open development process since 2015. ECMAScript is the standard specification of
JavaScript maintained by the Ecma Technical Committee 39 (TC39). In June 2015,
the committee introduced its new version called ECMAScript 6 (ES6,
2015)~\cite{es6}. It was the most extensive single update with new language
features, such as classes, arrow functions, promises, iterators, generators, and
proxies. Moreover,  to quickly adapt users' demands to the specification, TC39
members decided to release ECMAScript yearly and maintain it as an open-source
project in a GitHub repository\footnote{https://github.com/tc39/ecma262}.
Therefore, already six more versions of ECMAScript from ES7 to ES12 have been
released after ES6, and contributors have pushed \inred{2,150} commits to its
official GitHub repository.

However, most JavaScript engines did not support or partially supported newly
introduced features in the beginning. For example, even though ES6 was released
in June 2015, Google introduced Chrome 51, the first web browser that fully
supported ES6 features, in May 2016. Subsequently, Safari 10 (Apple, Sep. 2016)
and Firefox 54 (Mozilla, Sep. 2017) fully supported ES6 features. Thus,
client-side developers cannot use most ES6 features initially for the web
application development, although they wanted to utilize them for more concise
and readable JavaScript code.

Therefore, \textit{transpilers} showed up to fulfill developers' desires to use
new language features earlier. A transpiler is a source-to-source compiler that
takes a program written in a programming language and produces an equivalent
program in the same or a different programming language. For JavaScript, several
transpilers focused on how to \textit{desugar} newly introduced language
features in ES6+ to ES5. In 2011, Google started to develop a transpiler called
Traceur\footnote{https://github.com/google/traceur-compiler} for experimenting
with early ES6 features, and it provided a high-fidelity implementation of ES6
semantics. However, its severe speed degradation was unattractive to developers.
On the other hand, a 17-year-old developer introduced another transpiler
Babel\footnote{https://babeljs.io/} (originally named 6to5) to minimize runtime
overhead of desugared JavaScropt codes by sacrificing semantics preservation.
Nowadays, Babel has become an essential library to build JavaScript projects,
and it is downloaded about 32.4M times every
week\footnote{https://www.npmtrends.com/@babel/core}. Even though most engines
now support language features even in the latest ECMAScript, developers still
heavily utilize Babel to execute their JavaScript programs in legacy engines:
older versions of engines, legacy engines (e.g., Internet Explorer developed by
Microsoft), or regional web browsers (e.g., UC and QQ browsers in China) not
conforming to the latest ECMAScript.

\todo
