%% For double-blind review submission, w/o CCS and ACM Reference (max submission space)
\documentclass[sigplan,10pt,review,anonymous]{acmart}
\settopmatter{printfolios=true,printccs=false,printacmref=false}
%% For double-blind review submission, w/ CCS and ACM Reference
%\documentclass[sigplan,review,anonymous]{acmart}\settopmatter{printfolios=true}
%% For single-blind review submission, w/o CCS and ACM Reference (max submission space)
%\documentclass[sigplan,review]{acmart}\settopmatter{printfolios=true,printccs=false,printacmref=false}
%% For single-blind review submission, w/ CCS and ACM Reference
%\documentclass[sigplan,review]{acmart}\settopmatter{printfolios=true}
%% For final camera-ready submission, w/ required CCS and ACM Reference
%\documentclass[sigplan]{acmart}\settopmatter{}


%% Conference information
%% Supplied to authors by publisher for camera-ready submission;
%% use defaults for review submission.
\acmConference[PL'18]{ACM SIGPLAN Conference on Programming Languages}{January 01--03, 2018}{New York, NY, USA}
\acmYear{2018}
\acmISBN{} % \acmISBN{978-x-xxxx-xxxx-x/YY/MM}
\acmDOI{} % \acmDOI{10.1145/nnnnnnn.nnnnnnn}
\startPage{1}

%% Copyright information
%% Supplied to authors (based on authors' rights management selection;
%% see authors.acm.org) by publisher for camera-ready submission;
%% use 'none' for review submission.
\setcopyright{none}
%\setcopyright{acmcopyright}
%\setcopyright{acmlicensed}
%\setcopyright{rightsretained}
%\copyrightyear{2018}           %% If different from \acmYear

%% Bibliography style
\bibliographystyle{ACM-Reference-Format}
%% Citation style
%\citestyle{acmauthoryear}  %% For author/year citations
%\citestyle{acmnumeric}     %% For numeric citations
%\setcitestyle{nosort}      %% With 'acmnumeric', to disable automatic
                            %% sorting of references within a single citation;
                            %% e.g., \cite{Smith99,Carpenter05,Baker12}
                            %% rendered as [14,5,2] rather than [2,5,14].
%\setcitesyle{nocompress}   %% With 'acmnumeric', to disable automatic
                            %% compression of sequential references within a
                            %% single citation;
                            %% e.g., \cite{Baker12,Baker14,Baker16}
                            %% rendered as [2,3,4] rather than [2-4].


%%%%%%%%%%%%%%%%%%%%%%%%%%%%%%%%%%%%%%%%%%%%%%%%%%%%%%%%%%%%%%%%%%%%%%
%% Note: Authors migrating a paper from traditional SIGPLAN
%% proceedings format to PACMPL format must update the
%% '\documentclass' and topmatter commands above; see
%% 'acmart-pacmpl-template.tex'.
%%%%%%%%%%%%%%%%%%%%%%%%%%%%%%%%%%%%%%%%%%%%%%%%%%%%%%%%%%%%%%%%%%%%%%

%% macros with packages
% load packages
\usepackage{float}
\usepackage{amsmath,amsfonts}
\usepackage[ruled, vlined]{algorithm2e}
\usepackage{graphicx}
\usepackage{textcomp}
\usepackage{xcolor}
\usepackage{listings}
\usepackage{caption}
\usepackage{subcaption}
\usepackage{multirow}
\usepackage{booktabs}
\usepackage{makecell}
\usepackage{galois}
\usepackage{mathpartir}
\usepackage{bussproofs}
\usepackage{mathtools}
\usepackage{colortbl}
\usepackage{hhline}
\usepackage{stmaryrd}
\usepackage{microtype}
\usepackage{hyperref}

% basic
\newcommand{\inred}[1]{{\color{red}{#1}}}
\newcommand{\todo}{\inred{TODO}}

% tool name
\newcommand{\name}[1]{\textsf{#1}}
\newcommand{\sname}[1]{\name{\small #1}}
\newcommand{\stextbf}[1]{\textbf{\small #1}}
\newcommand{\jiset}{\sname{JISET}}
\newcommand{\lambdajs}{\lambda_\text{JS}}
\newcommand{\jscert}{\text{JSCert}}
\newcommand{\jsref}{\text{JSRef}}
\newcommand{\kjs}{\text{KJS}}
\newcommand{\javert}{\text{JaVerT}}
\newcommand{\jsil}{\text{JSIL}}

% JavaScript code style
\lstdefinelanguage{JavaScript}{
  keywords={async, await, break, case, catch, class, const, continue, debugger,
    default, delete, do, else, enum, export, extends, false, finally, for,
    function, if, import, in, of, instanceof, new, null, return, super, switch, this,
    throw, true, try, typeof, let, var, void, while, with, yield},
  keywordstyle=\color{blue}\bfseries,
  ndkeywordstyle=\color{darkgray}\bfseries,
  identifierstyle=\color{black},
  numberstyle=\tiny\color{darkgray},
  numbers=none,
  numbersep=5pt,
  sensitive=false,
  comment=[l]{//},
  morecomment=[s]{/*}{*/},
  commentstyle=\color{dkgreen},
  stringstyle=\color{red}\ttfamily,
  morestring=[b]',
  morestring=[b]",
  morestring=[b]`
}
\lstdefinestyle{JS}{
  language=JavaScript,
  extendedchars=true,
  basicstyle=\small\ttfamily,
  showstringspaces=false,
  showspaces=false,
  tabsize=2,
  breaklines=true,
  showtabs=false,
  captionpos=b
}

% codes
\newcommand{\code}[1]{\text{\lstinline[style=JS]!#1!}} 


%% start document
\begin{document}

%% Title information
\title[An Empirical Study of Desugaring in JavaScript]
{An Empirical Study of Desugaring in JavaScript}

%% Author information
%% Contents and number of authors suppressed with 'anonymous'.
%% Each author should be introduced by \author, followed by
%% \authornote (optional), \orcid (optional), \affiliation, and
%% \email.
%% An author may have multiple affiliations and/or emails; repeat the
%% appropriate command.
%% Many elements are not rendered, but should be provided for metadata
%% extraction tools.
\author{Jihyeok Park}
\orcid{0000-0001-8387-1984}
\affiliation{
  \department{School of Computing}
  \institution{KAIST}
  \city{Daejeon}
  \country{South Korea}
}
\email{jhpark0223@kaist.ac.kr}

\author{Dongjun Youn}
\orcid{0000-0002-5766-2035}
\affiliation{
  \department{School of Computing}
  \institution{KAIST}
  \city{Daejeon}
  \country{South Korea}
}
\email{f52985@kaist.ac.kr}

\author{Hyerin Park}
\orcid{\todo}
\affiliation{
  \department{School of Computing}
  \institution{KAIST}
  \city{Daejeon}
  \country{South Korea}
}
\email{hyerin.park@kaist.ac.kr}

\author{Sukyoung Ryu}
\orcid{0000-0002-0019-9772}
\affiliation{
  \department{School of Computing}
  \institution{KAIST}
  \city{Daejeon}
  \country{South Korea}
}
\email{sryu.cs@kaist.ac.kr}

%% Abstract
%% Note: \begin{abstract}...\end{abstract} environment must come
%% before \maketitle command
\begin{abstract}\label{sec:abstract}
  In 2015, ECMAScript 6 (ES6, 2015) introduced a new wave in JavaScript with
  various new language features. It is the start of the fast-evolving nature of
  JavaScript with an annual release cadence and open development process.
  However, most JavaScript engines did not support or partially supported newly
  introduced features in the beginning. Therefore, \textit{transpilers} showed
  up to fulfill developers' desires to use new language features before
  JavaScript engines would support them. They \textit{desugar} such features to
  transpile JavaScript programs from ES6 or later versions to ECMAScript 5 (ES5,
  2009). Even though most engines now support language features even in the
  latest ECMAScript, developers still heavily utilize them to execute their
  JavaScript programs in legacy engines. However, there is no empirical study to
  answer the correctness or quality of their desugaring process even though six
  years have passed since 2015.

  This paper takes the first step to systematically studying the desugaring
  process of Babel, the most dominating JavaScript transpiler. We measure the
  quality of desugaring process from ES6 or later versions to ES5 based on its
  \textit{semantics preservation}, \textit{speed degradation}, and \textit{code
  bloat}. We develop $\tool$, a JavaScript program generator using genetic
  algorithms, to automatically measure the quality and find the worst cases in
  each metric. To evenly treat each language feature under the same granularity,
  we introduce a solid definition of JavaScript language features and an
  automated approach to extract them from JavaScript programs using their
  statistical information. Our experiments show that Babel incorrectly desugars
  \inred{XX} language features, degrades the execution speed by \inred{XX.X}\%
  on average, and bloats code sizes by \inred{XX.X}\% on average. In the worst
  case, the speed was \inred{XX.X} times slower, and the size was \inred{XX.X}
  times increased. We believe that this study provides a foundation for building
  a sound and practical desugaring process in JavaScript transpilers.
\end{abstract}


% TODO in camera-ready
%% 2012 ACM Computing Classification System (CSS) concepts
%% Generate at 'http://dl.acm.org/ccs/ccs.cfm'.
% \begin{CCSXML}
% <ccs2012>
% <concept>
% <concept_id>10011007.10011006.10011008</concept_id>
% <concept_desc>Software and its engineering~General programming languages</concept_desc>
% <concept_significance>500</concept_significance>
% </concept>
% <concept>
% <concept_id>10003456.10003457.10003521.10003525</concept_id>
% <concept_desc>Social and professional topics~History of programming languages</concept_desc>
% <concept_significance>300</concept_significance>
% </concept>
% </ccs2012>
% \end{CCSXML}
% 
% \ccsdesc[500]{Software and its engineering~General programming languages}
% \ccsdesc[300]{Social and professional topics~History of programming languages}
%% End of generated code

%% Keywords
%% comma separated list
\keywords{\todo}

%% \maketitle
%% Note: \maketitle command must come after title commands, author
%% commands, abstract environment, Computing Classification System
%% environment and commands, and keywords command.
\maketitle

%% body of the paper
\section{Introduction}\label{sec:intro}

\todo

\section{Related Work}\label{sec:related}

\begin{itemize}
  \item JavaScript specificaiton: ECMAScript (ES12, 2021)~\cite{es12}
  \item JavaScript tests: Test262~\cite{test262}
  \item JavaScript tools: engines~\cite{v8, jscore, chakra, spidermonkey},
    static analyzers~\cite{safe, safe2, tajs, wala, jsai},
    debugger~\cite{jsexplain}, verification tools~\cite{javert, javert2,
    ad-safety, javanni}, symbolic execution~\cite{symbolic-js, sym-js, expo-se},
    concolic testing~\cite{jalangi, type-conc-test}.
  \item Desugaring in JavaScript: $\lambdajs$~\cite{lambda-js}, S5~\cite{s5},
    module system rewriter~\cite{js-module}, \code{with} statements
    rewriter~\cite{js-with}
  \item JISET family: JISET~\cite{jiset}, JEST~\cite{jest}, JSTAR~\cite{jstar}.
\end{itemize}

\todo

\section{Conclusion}\label{sec:conclusion}

\todo


% TODO in camera-ready
%% Acknowledgments
% \begin{acks}
% \end{acks}

%% Bibliography
\bibliography{ref}

\end{document}
